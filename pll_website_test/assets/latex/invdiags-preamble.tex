
% cut-down preamble + macros from Kapulkin–Lumsdaine “CwAs of inverse diagrams”, for compiline standalone figures

%% Submission vs arXiv verions
\newcommand{\arxivorsubmission}[2]{\ifarxiv #1\else #2\fi}
\newcommand{\arxivonly}[1]{\arxivorsubmission{#1}{}}
\newcommand{\submissiononly}[1]{\arxivorsubmission{}{#1}}

%% Input handling:
\usepackage[utf8]{inputenc}  % to allow unicode in source

%% AMS and other general math packages:
\usepackage{amsmath}
\usepackage{amssymb}
\usepackage{amsthm}
\usepackage[mathscr]{eucal}

%% General style packages:
\usepackage{xcolor}
\definecolor{darkgreen}{rgb}{0,0.45,0}
\definecolor{darkred}{rgb}{0.75,0,0}
\definecolor{darkblue}{rgb}{0,0,0.6}
\usepackage[linktoc=all,colorlinks,citecolor=darkgreen,linkcolor=darkred,urlcolor=darkblue]{hyperref}
\usepackage{enumerate} % for customising enumerated lists

%% For rules of type theory:
\usepackage{mathpartir}

%% Graphics and diagrams packages:
\usepackage{graphicx}
\usepackage{tikz}
\usepackage{tikz-cd}
%\usepackage{xypic} % prefer tikz-cd if possible
\usepackage{mathtools} % for \mathclap, used to center extra-wide diagrams.
\usepackage{adjustbox} % for scaling down over-wide diagrams

%% Misc packages
\usepackage{xifthen} % for \ifthenelse
\usepackage{ifmtarg} % check if macro arguments are empty



%%% General style macros (non-mathematical)

\newcommand{\todo}[1]{} % {\textcolor{red}{\textbf{#1}}}
\newcommand{\optionaltodo}[1]{} % {\textcolor{red!40!blue}{#1}}

\newcommand{\defemph}[1]{\textbf{#1}} % for highlighting terms introduced/defined

%%% For scaling over-wide diagrams

\newenvironment{maxwidth}{\begin{adjustbox}{max width=\linewidth}}{\end{adjustbox}}

\newenvironment{maxwidthtikzpicture}[1][]
  {\begin{maxwidth}\begin{tikzpicture}[#1]}
  {\end{tikzpicture}\end{maxwidth}}

\newenvironment{maxwidthtikzcd}[1][]
  {\begin{maxwidth}\begin{tikzcd}[#1]}
  {\end{tikzcd}\end{maxwidth}}


%%% TikZ setup and style

%% Library imports
\usetikzlibrary{calc}
\usetikzlibrary{fit}
\usetikzlibrary{shapes}
\usetikzlibrary{arrows}
\usetikzlibrary{decorations.markings}
\usetikzlibrary{decorations.pathmorphing}
\usetikzlibrary{positioning}
\usetikzlibrary{intersections}

%% Set default styles for tikz-cd

\tikzset{
  commutative diagrams/arrow style=tikz,
  commutative diagrams/diagrams={row sep=large},
}

%% For using tikz-cd style in \tikzpicture

\tikzset{cd-style/.style={commutative diagrams/every diagram}}
\tikzset{cd-arrow-style/.style={commutative diagrams/.cd, every arrow, every label}}
\newcommand{\arr}[1][]{\draw[cd-arrow-style,#1]}

% clone of how tikz-cd defines arrow labels,
% allowing them to be used uniformly in tikz-cd environment and in general tikz pictures:
\makeatletter
\tikzset{
  label/.style n args={2}{
    edge node={node [
      execute at begin node=\iftikzcd@mathmode$\fi,%$
      execute at end node=\iftikzcd@mathmode$\fi,%$
      /tikz/commutative diagrams/.cd,every label,
      #2
      ] {#1}}
  }
}
\makeatother

%% Custom arrow tips & aliases

\tikzset{fibtip/.tip={Triangle[open,angle=60:4.5pt]}}
\tikzset{tfibtip/.tip={Bar[sep]Triangle[open,angle=45:4pt]}} % Based on http://tex.stackexchange.com/a/150213

% Alternatives, for xypic-style arrowheads:
% \tikzset{fibtip/.tip={xytri}}
% \tikzset{tfibtip/.tip={Bar[sep] xytri}}

\pgfarrowsdeclaredouble{fibbtip}{fibbtip}{fibtip}{.fibtip}

\pgfarrowsdeclarealias{c}{c}{left hook}{right hook}
\pgfarrowsdeclarealias{c'}{c'}{right hook}{left hook}

% for indicating pullbacks
\tikzset{drpb/.style={commutative diagrams/.cd, dr, phantom, "\lrcorner", very near start}}

%% Semantic arrow styles

\tikzset{fib/.code={\pgfsetarrowsend{fibtip}}}
\tikzset{fibb/.code={\pgfsetarrowsend{fibbtip}}}
\tikzset{tfib/.code={\pgfsetarrowsend{tfibtip}}}
\tikzset{inj/.code={\pgfsetarrowsstart{c}}}
\tikzset{tcof/.style={tail}}
\tikzset{zigzag/.style={commutative diagrams/rightsquigarrow}}
\tikzset{lw/.style={"lw"{#1}},lw/.default={very near end}}
\tikzset{lw'/.style={"lw"'{#1}},lw'/.default={very near end}}
\tikzset{weq/.style={"\sim"}}
\tikzset{weq'/.style={"\sim"'}}
\tikzset{sup/.style={label={#1}{auto=left,pos=1}}}  % TODO: improve placement!
\tikzset{sub/.style={label={#1}{auto=right,pos=1}}}

%% Inline arrows

% TikZ inline arrows, adapted from http://tex.stackexchange.com/a/116324 and http://tex.stackexchange.com/a/188351

\newcommand{\generalto}[2]{ \mathrel{\mkern-1mu
  \tikz[baseline={([yshift=-0.55ex]a.south)}]{%
    \node[minimum width=1.5em,align=center,inner xsep=0.5ex,inner ysep=0.15ex] (a) {$\scriptstyle #2$};
    \draw[#1] (a.south west) -- (a.south east);}
 \mkern-1mu}}

\newcommand{\generalfrom}[2]{ \mathrel{\mkern-1mu
  \tikz[baseline={([yshift=-0.55ex]a.south)}]{%
    \node[minimum width=1.5em,align=center,inner xsep=0.5ex,inner ysep=0.15ex] (a) {$\scriptstyle #2$};
    \draw[#1] (a.south east) -- (a.south west);}
  \mkern-1mu}}

\renewcommand{\to}[1][]{ \generalto{->}{#1} }
\newcommand{\from}[1][]{ \generalfrom{->}{#1} }
\newcommand{\fibto}[1][]{ \generalto{fib}{#1} }
\newcommand{\fibbto}[1][]{ \generalto{fibb}{#1} }
\newcommand{\extto}[1][]{ \generalto{>->}{#1} }
\newcommand{\injto}[1][]{ \generalto{->,inj}{#1} }
\newcommand{\lwto}[1][]{ \generalto{->}{#1}_{\lw} }

\makeatletter
\newcommand{\weqto}[1][]{\@ifmtarg{#1}{\generalto{->}{\sim}}{\NotYetDefined}}
  % note: with non-empty argument, should put the \sim below the arrow; but not clear to me (PLL) how to do this using \generalto.
\newcommand{\weqfrom}[1][]{\@ifmtarg{#1}{\generalfrom{->}{\sim}}{\NotYetDefined}}
  % note: with non-empty argument, should put the \sim below the arrow; but not clear to me (PLL) how to do this using \generalto.
\makeatother
\newcommand{\zigzagto}[1][]{ \generalto{zigzag}{#1} }

% TODO: improve, consistentise with the rest?
\newcommand{\parpair}[2]{\begin{tikzcd}[ampersand replacement=\&] #1 \ar[shift left]{r} \ar[shift right]{r} \& #2 \end{tikzcd}}


% vertical alignment for qed after diagram, from https://tex.stackexchange.com/a/195443/
\newenvironment{verticalhack}
  {\begin{array}[b]{@{}c@{}}\displaystyle}
  {\\\noalign{\hrule height0pt}\end{array}}

%% Letter-like macros:
\newcommand{\A}{\mathbf{A}}
\newcommand{\Abar}{\bar{A}}
\renewcommand{\AA}{{\vec A}}
\newcommand{\B}{\mathbf{B}}
\newcommand{\BB}{{\vec B}}
\newcommand{\Bbar}{\bar{B}}
\renewcommand{\c}{c}
\newcommand{\C}{\mathbf{C}}
\newcommand{\catC}{\mathscr{C}}
\newcommand{\CC}{{\vec C}}
\newcommand{\D}{\mathbf{D}}
\newcommand{\E}{\mathbf{E}}
\newcommand{\e}{\varepsilon}
\newcommand{\catE}{\mathscr{E}}
\newcommand{\EE}{{\vec E}}
\newcommand{\ff}{\vec f}
\newcommand{\h}{\vec{h}}
\newcommand{\I}{\mathcal{I}}
\newcommand{\J}{\mathcal{J}}
\renewcommand{\L}{\mathcal{L}}
\newcommand{\R}{\mathcal{R}}
\newcommand{\N}{\mathbb{N}}
\newcommand{\T}{\mathbf{T}}
\newcommand{\W}{\mathcal{W}}
\newcommand{\x}{\vec x}
\newcommand{\X}{\mathbf{X}}
\newcommand{\Xbar}{\overline{X}}
\newcommand{\y}{\vec y}
\newcommand{\Y}{\mathbf{Y}}

% Several options for the yoneda embedding:
% \font\maljapanese=dmjhira at 2ex % you can change this "2ex" value
%\newcommand{\yon}{\textrm{\maljapanese\char"48}}
\newcommand{\yon}{\mathbf{y}}
%\renewcommand{\yon}[1]{{#1 \mathord{\downarrow}}}
\newcommand{\del}{\partial}
\newcommand{\bdry}[1]{{\del #1}}

\newcommand{\GGamma}{\vec \Gamma}
\newcommand{\DDelta}{\vec \Delta}
\newcommand{\vectau}{\vec \tau}

% Classes of maps in WFS’s
\newcommand{\Anod}{\mathcal{A}}
\newcommand{\Fib}{\mathcal{F}}
\newcommand{\TFib}{\mathcal{TF}}
\newcommand{\Cof}{\mathcal{C}}
\newcommand{\TCof}{\mathcal{TC}}
\newcommand{\WEq}{\mathcal{W}}

\newcommand{\cell}[1]{{#1}^\textit{cell}}


%% Standard simplicial and semi-simplicial sets
%\newcommand{\semi}{\mathrm{i}}
%\newcommand{\simp}[1]{{\Delta [#1]} }
%\newcommand{\simpi}[1]{{\Delta_\semi [#1]} }
%\newcommand{\bdry}[1]{{\delta \simp{#1}}}
%\newcommand{\bdryi}[1]{{\delta \simpi{#1}}}
%\newcommand{\horn}[2]{{\Lambda^{#1} [#2]} }
%\newcommand{\horni}[2]{{\Lambda^{#1}_\semi [#2]} }

%% Category names
\newcommand{\Cat}{\mathrm{Cat}}
\newcommand{\sSet}{\mathrm{sSet}}
\newcommand{\ssSets}{\mathrm{ssSet}}
\newcommand{\CxlCat}{\mathrm{CxlCat}}
\newcommand{\Lex}{\mathrm{Lex}}
\newcommand{\LCCC}{\mathrm{LCCC}}
\newcommand{\CwA}{\mathrm{CwA}}
\newcommand{\CompCat}{\mathrm{CompCat}} % TODO: remove this, to make sure we never use it.
\newcommand{\Mod}{\mathrm{Mod}}
\newcommand{\Cl}{\operatorname{Cl}}
\newcommand{\weCat}{\mathrm{weCat}}
\newcommand{\ElTop}{\mathrm{ElTop}}
\newcommand{\Set}{\mathrm{Set}}
\newcommand{\FinSet}{\mathrm{FinSet}}
\newcommand{\FibCat}{\mathrm{FibCat}}

\newcommand{\SpanCat}{\mathrm{Span}}
\newcommand{\WkMapCat}{\mathrm{WkMap}}
\newcommand{\EqvCat}{\mathrm{Eqv}}
\newcommand{\EqvReflCat}{\mathrm{EqvRefl}}
\newcommand{\EqvCompCat}{\mathrm{EqvComp}}

\newcommand{\supfunctor}[2]{\ifthenelse{\equal{#2}{}}%
  {(-)^\mathrm{#1}}%
  {{#2}^\mathrm{#1}}%
}
\newcommand{\Span}[1][]{\supfunctor{Span}{#1}}
\newcommand{\Eqv}[1][]{\supfunctor{Eqv}{#1}}
\newcommand{\EqvRefl}[1][]{\supfunctor{EqvRefl}{#1}}
\newcommand{\EqvComp}[1][]{\supfunctor{EqvComp}{#1}}

\newcommand{\Eqvold}{\mathrm{Eqv}}
\newcommand{\Spanold}{\mathrm{Span}}

\newcommand{\Nf}{\mathrm{N}_{\mathrm{f}}}

\newcommand{\elem}[2][]{\int_{#1} #2} % category of elements of a functor

\newcommand{\freeCxlCat}[2][]{\freestrux{#2}{#1}{}}
\newcommand{\freeCwA}[2][]{\freestrux{#2}{#1}{\CwA}}

\newcommand{\fibslice}[2]{{#1}/\!/{#2}} % fibrant slices of CwAs/Cxl Cats
\newcommand{\fibslicefunc}[2]{{#1}/\!/{#2}} % fibrant slices of functors between them.
  % Probably the same as \fibslice, unless we use just C(Γ) as the notation for that.

\newcommand{\slice}[2]{( #2 \downarrow #1 ) }
\newcommand{\strictslice}[2]{\partial \slice{#1}{#2} }

\newcommand{\cotensor}[2]{#1 \mathbin{\hat{\pitchfork}} #2}

%% Word-like operators
% Note: most such operators, including anything that may be applied without brackets, eg “ob C”, should be defined with \DeclareMathOperator to get good spacing.
% On the other hand, macros which should explicitly *not* be spaced as operators (e.g. \id), or which are primarily used on their own in super/subscripts (eg \op) should not be given this way.
\DeclareMathOperator{\Hom}{Hom}
\DeclareMathOperator{\Surj}{Surj}
\DeclareMathOperator{\Map}{Map}
\DeclareMathOperator{\Cyl}{Cyl}
\newcommand{\op}{\mathrm{op}}
\DeclareMathOperator{\ob}{Ob}
\DeclareMathOperator{\ft}{ft}
\DeclareMathOperator{\Ho}{Ho}
\newcommand{\id}{\mathrm{id}}
\newcommand{\ev}{\mathsf{ev}}
\newcommand{\lw}{\mathrm{lw}} % “levelwise”
\newcommand{\str}{\mathrm{str}} % strictification functor
\DeclareMathOperator{\core}{core}
\DeclareMathOperator{\cod}{cod}
\DeclareMathOperator{\dom}{dom}
\newcommand{\wkreedy}{\mathrm{wR}}
\newcommand{\reedy}{\mathrm{R}}
\DeclareMathOperator{\im}{im}

%% Binary relations
\newcommand{\Iff}{\Leftrightarrow}
\newcommand{\Imp}{\Rightarrow}
\newcommand{\Pmi}{\Leftarrow}
\newcommand{\iso}{\cong}
\renewcommand{\equiv}{\simeq}
\newcommand{\rhomot}{\sim_r}
\newcommand{\ideq}{=_{\Id}}
\newcommand{\wrhomot}{\sim}
\newcommand{\homot}{\sim}
\newcommand{\orthog}{\pitchfork}

%% Syntax of type theory:
% judgements
\newcommand{\of}[1][1]{\mspace{#1 mu plus 1.0mu}\mathord{:}\mspace{#1 mu plus 1mu}}
  % usage: for variable declarations, gives a tighter colon than the default
  % example: [ x \of \R, n \of \N \types x^n : \R ]
  % usually [ x \of A ] is good; for wider (e.g. when types are complex), use \of[2], \of[3], etc.
\newcommand{\types}{\vdash}
\newcommand{\type}{\ \textsf{type}}
\newcommand{\cxt}{\ \textsf{cxt}}
\newcommand{\emptycxt}{[\ ]}

% logical primitives (and abbreviations)
\newcommand{\Id}{\mathsf{Id}}
\newcommand{\refl}{\mathsf{r}}
\newcommand{\unit}{\mathsf{1}}
\newcommand{\synSigma}{\mathsf{\Sigma}}
\newcommand{\synPi}{\mathsf{\Pi}}
\newcommand{\Piext}{\mathsf{\Pi}_{\mathsf{ext}}}
\newcommand{\synsplit}{\mathsf{split}}
\newcommand{\synJ}{\mathsf{J}}
%\newcommand{\pair}[1][-,-]{\langle #1 \rangle} % TODO: perhaps unify with others?
\newcommand{\pair}{\mathsf{pair}}
\newcommand{\rec}{\mathsf{rec}} % for unit type
\newcommand{\funext}{\mathsf{funext}}
\newcommand{\funextcomp}{\mathsf{funext\textsf{-}comp\textsf{-}prop}}
\newcommand{\HoTT}{\mathsf{HoTT}}

% types of structure on CwA’s
\newcommand{\ext}{\mathrm{ext}}

% defined notions
\newcommand{\IsEquiv}{\mathsf{IsEquiv}}
\newcommand{\ap}{\mathsf{ap}}

% For contextual categories and CwA’s:
\newcommand{\Ty}{\mathrm{Ty}}
\newcommand{\Tm}{\mathrm{Tm}}
\newcommand{\cxtext}[1]{{#1}^{\ast}}

%% Miscellaneous operations
\newcommand{\lorth}[1]{{}^{\orthog}{#1}}
\newcommand{\rorth}[1]{{#1}^{\orthog}}

\newcommand{\aangled}[1]{\langle\!\langle\, #1 \,\rangle\!\rangle} % TODO: find better name?

\newcommand{\freestrux}[3]{\aangled{#1}_{#2}^{#3}}

\newcommand{\barbar}[1]{\overline{\overline{#1}}}

\newcommand{\restr}[1]{|_{#1}} % restriction

\newcommand{\descend}[2]{#1\! \urcorner \, #2} % descent of a type along an elim strux

\newcommand{\disc}[1]{{#1}^{\circ}}
\newcommand{\binc}[1]{d_{#1}}
\newcommand{\copsh}[1]{\Set^{#1}}
\newcommand{\extend}[2]{#1 \triangleleft #2}

\newcommand{\lscott}{[\![}
\newcommand{\rscott}{]\!]}
